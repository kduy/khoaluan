\chapter{Bài toán dự báo trên dữ liệu Telecom}
\ifpdf
    \graphicspath{{3_experiment/figures/PNG/}{3_experiment/figures/PDF/}{3_experiment/figures/}}
\else
    \graphicspath{{3_experiment/figures/EPS/}{3_experiment/figures/}}
\fi

\section{Dữ liệu telecom và bài toán dự báo lưu lượng}
Trong lĩnh vực viễn thông, \textbf{lưu lượng} là thông tin (người sử dụng/báo hiệu) mang trên các kênh truyền dẫn trên mạng viễn thông. Chất lượng dịch vụ mạng phụ thuộc rất lớn vào việc lưu lượng thông tin trên mạng có được thông suốt hay không. Trong bối cảnh, cơ sở vật chất một số mạng chưa theo kịp với tốc độ phát triển bùng nổ các số thuê bao ở một số khu vực. Tình trạng này có thể dẫn đến việc xảy ra nghẽn mạng, dịch vụ mạng chập chờn, kết nối kém...đặc biệt là trong giờ cao điểm (hay còn gọi là giờ bận). Đó một phần do không dự báo được trước chính xác cường độ lưu lượng trên kênh truyền tại thời điểm đó để phân bổ và có phương án hợp lý. Một số khu vực thì kênh truyền khá rỗi rãi và không hoạt động hết công suất, trong khi một số khu vực khác thì nghẽn do lưu lượng quá lớn. Dự báo được thời điểm, lưu lượng tại một khu vực sẽ giúp nhà mạng phân bổ lại cơ sở hạ tầng và chia sẻ kênh truyền giúp cải thiện dịch vụ tốt hơn. Bài toán đặt ra ở đây là từ dữ liệu các cuộc gọi, tin nhắn và các dịch vụ giá trị gia tăng khác (call detail record) làm thế nào để ta có thể xác định tương đối lưu lượng kênh truyền thời điểm hiện tại và dự đoán lưu lượng trọng tương lai. 

Ta tập trung vào phân tích lưu lượng tại các \textbf{giờ bận}. Đó là giờ có lưu lượng trao đổi qua các kênh là lớn nhất trong ngày. Lưu lượng trong giờ bận được xác định theo công thức Erlangs \citep{flo95}:
\begin{equation}
A = C*t / T
\end{equation}
với $C$ là số cuộc gọi trong giờ bận, $t$ là thời gian trung bình của một cuộc gọi và $T$ là thời gian khảo sát. 
Trong thực tế, không chỉ có các cuộc gọi được thực hiện mà còn có các tin nhắn và các dịch vụ giá trị gia tăng khác như tải nhạc chuông, nghe nhạc... Công thức 3.1 có thể được mở rộng cho phù hợp với dữ liệu như sau:
\begin{equation}
A' = (\sum\limits(n_{sms}) + \sum\limits{callDuration})/T
\end{equation}
hay nói cách khác, lưu lượng trong giờ bận được tính bằng tổng số tin nhắn và tổng thời gian gọi hoặc sử dụng dịch vụ giá trị gia tăng.


\textbf{Dữ liệu phân tích}


Thực nghiệm sử dụng dữ liệu thu thập thông tin các cuộc gọi, tin nhắn, dịch vụ của người dùng mạng di động BeeLine từ 15/05/2010 đến 14/11/2010. Bản ghi chi tiết cuộc gọi gồm rất nhiều trường dữ liệu nhưng ta quan tâm đến một số trường quan trọng:
\begin{description}
\item[$Rate-DT$:] thời điểm bắt đầu cuộc gọi, tin nhắn, dịch vụ
\item[$Trans-DT$:] thời điểm kết thúc cuộc gọi, dịch vụ
\item[$PRIMARY-UNIT$:] đo mức độ sử dụng dịch vụ, cuộc gọi... Với tin nhắn thì $PRIMARY-UNIT=1$, đối với đại lượng này tính bằng thời gian đàm thoại, đối với một số dịch vụ gia tăng thì có thể tính bằng lượng tải về Kb, Mb, Gb...
\end{description}
Sau khi thống kê xác định các giờ bận trong ngày thì ta có kết quả như Hình~\ref{1_h_h}
\figuremacroW{1_h_h}{Thống kê lưu lượng theo giờ}{}{0.7}
Dựa vào biểu đồ, ta chọn khoảng thời gian từ 8h sáng đến 12h đêm là khoảng giờ bận. Tính tổng lưu lượng của các giờ bận trong ngày 
ta có kết quả như Hình
\figuremacroW{3_luuluong}{Lưu lương các kênh truyền tính theo ngày}{Đơn vị: 1 triệu Elangs}{0.7}
Tập dữ liệu là chuỗi thời gian gồm 184 phần tử. Để tiến hành thực nghiệm độ hiệu quả của mô hình ta chia tập theo tỉ lệ $training : test = 160 : 24$. Trong quá trình thực nghiệm, ta xây dựng mô hình dựa trên dữ liệu 160 ngày đầu tiên và dự đoán kết quả của 24 ngày tiếp theo. Trong hình~\ref{3_train_test}, đồ thị bên trái đường nét đứt mô tả tập học còn phần bên phải đường nét đứt mô tả tập kiểm tra mà ta sẽ sử dụng để đối chiếu với kết quả dự báo. Mô hình được ứng dụng thử nghiệm là mô hình ARIMA và Mạng nơ ron nhân tạo được nói tới trong chương 2.
\section{Thực nghiệm với ARIMA}
\subsection{Xác định mô hình}
Bước đầu tiên là quan sát đồ thị của chuỗi thời gian(Hình~\ref{3_train}). Ta nhận xét, chuỗi này đang có xu hướng giảm dần và có thể xuất hiện tính chu kì thời vụ với chu kì là $7$ (ứng với 7 ngày trong một tuần lễ).
\figuremacroW{3_train}{Lưu lượng kênh truyền từ 15/05/2010 đến 14/11/2010 của mạng di động Beeline}{}{0.7}

Bước tiếp theo là kiểm tra độ ổn định của phương sai hay nói cách khác là kiểm tra sự thay đổi mức độ biến thiên quanh giá trị trung bình của chuỗi theo thời gian. Trong trường hợp, phương sai không ổn định, ta cần phải biến đổi dữ liệu. Đối với dữ liệu hiện tại, dựa vào hình~\ref{3_boxcox_train}, với độ chính xác 95\%, giá trị hợp lý logarit cực đại có thể đạt được nếu $\lambda \in [0,1]$. Khi $\lambda = 1$, theo Phần 2.2.2.1, dữ liệu không cần phải biến đổi. Ta sẽ chọn phương án không biến đổi dữ liệu, nếu mô hình đạt được cho kết quả dự đoán không tốt, việc biến đổi dữ liệu có thể thực hiện ở các vòng lặp sau.
\figuremacroW{3_boxcox_train}{Đồ thị hàm hợp lý loga}{}{0.7}

Bước thứ 3 trong pha xác định mô hình là kiểm tra tính dừng của chuỗi thời gian.
Ta nhận thấy đồ thị hàm ACF (Hình~\ref{3_acf}) của chuỗi thời gian lúc đầu có giá trị lớn ở đầu rồi sau đó giảm dần đều. Chứng tỏ đây có thể là chuỗi không có tính chất dừng.
\figuremacroW{3_acf}{Đồ thị hàm ACF}{}{0.5}
Để khẳng định chắc chắn, ta sử dụng phương pháp kiểm tra \textit{Augmented Dicky-Fuller} trên dữ liệu và thu được kết quả: 
\begin{itemize}
\item giá trị thống kê $statistic = -2.0188$
\item $p-value = 0.3068$
\end{itemize}  
Giá trị $p-value > 0.05$ khá lớn, do đó ta chấp nhận giải thuyết $Ho$ cho rằng chuỗi thời gian đang kiểm tra không phải là chuỗi dừng. Ta cần phải thực hiện sai phân cấp 1 trên chuỗi thời gian. 

Kiểm tra \textit{Augmented Dicky-Fuller} với chuỗi mới thu được sau khi sai phân, ta có kết quả
\begin{itemize}
\item giá trị thống kê $statistic = -9.7984$
\item $p-value = 0.01$
\end{itemize} 
Giá trị $p-value < 0.05$, ta có thể bác bỏ giả thuyết $H_0$. Do đó, chuỗi mới là chuỗi có tính chất dừng(Hình~\ref{3_tpdif}).
\figuremacroW{3_tpdif}{Đồ thị chuỗi đã được sai phân 1 lần}{}{0.5}

Quan sát đồ thị hàm ACF (Hình ~\ref{3_acf_dif})và PACF (Hình~\ref{3_pacf_dif})của chuỗi thời gian sau khi đã sai phân ta thấy tại các độ trễ 7 , 14, 21... giá trị tăng cao hơn so với các giá trị xung quanh. Dự đoán chắc chắn hơn chu kỳ sẽ là $lags=7$. 
\figuremacroW{3_acf_dif}{Đồ thị hàm ACF của chuỗi đã được sai phân}{}{0.5}
\figuremacroW{3_pacf_dif}{Đồ thị hàm PACF của chuỗi đã được sai phân}{}{0.5}

Tại $lag =14$(là lags trội trong chu kì thứ 2) có giá trị vượt trội và quá ngưỡng sai số cho phép, dự đoán thành phần $P, Q$ của tính chu kì mùa vụ có giá trị  $P_7=Q_7=2$. Tại $lag = 2$ cả giá trị acf và pacf đều có giá trị trội hơn xung quanh, dự đoán đây có thể là mô hình $ARIMA(2,1,0)$ với thành phần chu kì mùa $(2,0,2)_7$ hoặc $ARIMA(0,1,2)$ với thành phần chu kì mùa $(2,0,2)_7$. 

Nếu áp dụng khử \textit{sai phân chu kỳ} với chu kì $k=7$ cho chuỗi $\Delta{x_t}$
\begin{equation}
 \Delta_7(\Delta(x_t)) = (1-B^7)\Delta{x_t} = \Delta{x_t}-\Delta{x_{t-7}} = x_t-x_{t-1} - x_{t-7}+x_{t-8}
\end{equation}
Ta thu được đồ thị hàm ACF và hàm PACF của chuỗi mới($\Delta_7(\Delta(x_t))$) như sau
\figuremacroW{3_acf_difdif}{Đồ thị hàm ACF của chuỗi $\Delta_7(\Delta(x_t))$}{}{0.5}
\figuremacroW{3_pacf_difdif}{Đồ thị hàm PACF của chuỗi $\Delta_7(\Delta(x_t))$}{}{0.5}

Quan sát Hình~\ref{3_acf_difdif} và~\ref{3_pacf_difdif}, ta nhận thấy giá trị ACF và PACF của chuỗi $\Delta_7(\Delta(x_t))$ có giá trị rất trội tại điểm trễ $lag = 7$. Do vậy, ta đề xuất thêm mô hình có thành phần chu kì thời vụ $(1,1,1)_7$, $(0,1,1)_7$, $(1,1,0)_7$

Do số lượng mô hình đề xuất khá lớn, ta sử dụng chỉ số sai số AIC(the Akaike Information Criterion)(Bảng~\ref{aic}) để chọn mô hình có AIC nhỏ hơn.
\begin{table}[htbp]
  \centering
  \caption{Độ đo AIC của các mô hình đề xuất}
    \begin{tabular}{ccc}
          & \textbf{(0,1,2)} & \textbf{(2,1,0)} \\
\hline
    $(2,0,0)_7$ & 1029.12 & 1029.37 \\
    $(0,0,2)_7$ & 1036.33 & 1036.72 \\
    $(2,0,2)_7$ & 1021.29 & 1021.48 \\
    $(1,1,0)_7$ & 1011.63 & 1011.58 \\
    $(0,1,1)_7$ & 990.39 & 990.41 \\
    $(1,1,1)_7$ & 989.01 & 989.06 \\
    \end{tabular}%
  \label{aic}%
\end{table}%

Nhận thấy mô hình $ARIMA (0,1,2)(1,1,1)_7$ có chỉ số AIC nhỏ nhất. Đầu tiên, ta tiến hành tiếp tục ước lượng tham số và kiểm định mô hình này. Nếu trong trường hợp mô hình này không thỏa mãn, thử nghiệm sẽ được tiếp tục với các mô hình còn lại theo độ ưu tiên mô hình có chỉ số AIC nhỏ hơn. Kết quả thực nghiệm cho thấy, mô hình $ARIMA (0,1,2)(2,0,2)_7$ cho kết quả về tính chất nhiễu và sai số mô hình tốt nhất. Quá trình ước lượng và kiểm định tiếp theo sẽ được thực hiện với mô hình này.

\subsection{Ước lượng tham số}
Sử dụng ước lượng tham số cực đại, ta thu được các tham số 
\begin{itemize}
	\item Thành phần trung bình trượt MA(2): $ma1 = -0.0510$, $ma2 = -0.1344$
	\item Thành phần tự hồi quy chu kỳ SAR(1): $sar1 = -0.0795$, $sar2 = 0.9177 $
	\item Thành phần trung bình trượt chu kỳ SMA(2): $sma1 = 0.2494$, $sma2 = -0.7136 $
\end{itemize}

\subsection{Kiểm định mô hình}
\begin{enumerate}
\item Kiểm tra \textit{Shapiro-Wilk} cho sai số chuẩn hóa: Kết quả đo được độ chuẩn tắc của sai số là $W=0.991$ với $p=0.4 > 0.05$. Do giá trị $p-value$ khá lớn, nên ta chấp nhận giả thuyết $H_0$ cho rằng sai số chuẩn hóa có tính độc lập cao.
\item Kiểm tra \textit{runs test} cho sai số chuẩn hóa: Kết quả thu được $p-value = 0.997$. Tương tự, ta kết luận được chuỗi sai số chuẩn hóa có tính độc lập cao
\item Kiểm tra \textbf{Ljung-Box}: Kết quả thu được $p-value = 0.8463$ rất cao nên chấp nhận giả thuyết $H_0$ cho rằng chuỗi sai số là chuỗi nhiễu trắng Gaussian. Ngoài ra $Q_{*} = 4.8607$,$ \chi^{2}_{0.95,9} = 16.91798$ nên $Q_{*}<\chi^{2}_{0.95,9}$ , ta chấp nhận giả thuyết cho rằng đây là mô hình ARIMA phù hợp
\end{enumerate}
\subsection{Dự báo}
Kết quả dự báo cho 23 ngày tiếp theo được mô tả ở Hình~~\ref{3_pr_arima}
\figuremacroW{3_pr_arima}{So sánh kết quả dự đoán 23 ngày tiếp theo sử dụng mô hình $ARIMA(0,1,2)(2,0,2)_7$}{(đường dự báo arima - nét đứt,đường thực tế - nét liền)}{1}
Mô hình dự báo khá chính xác tại một số điểm như ngày thứ 3, 9, 10, 14, 15. Ngoài ra, do tính chất không ổn định của dữ liệu nên dự đoán ở một số ngày vẫn có sai số đáng kể. Theo như xu hướng thì chuỗi có xu hướng đi xuống nên dự báo cũng chính xác theo hướng như vậy. Ta thấy ngoài một số điểm bất thường thì ARIMA dự báo khá chính xác mục tiêu ngắn hạn nhưng để đạt đến mức độ hoàn thiện hơn thì cần quay trở lại xác định lại mô hình. Việc này đòi hỏi cần kinh nghiệm và tốn rất nhiều công sức.

\section{Thực nghiệm với mạng nơ ron nhân tạo}
Trong mô hình mạng nơ ron nhân tạo với cả hai kĩ thuật huấn luyện lan truyền ngược và giải thuật di truyền, ta tiến hành thực nghiệm với mô hình đơn giản có 3 phần tử lớp đầu vào, phần tử lớp ẩn nằm trong khoảng từ $2$ đến $9$ và 1 phần tử lớp đầu ra. Như vậy, giá trị $x_t$ sẽ được dự báo thông qua đầu vào là giá trị của 3 phần tử liền trước nó trong chuỗi thời gian $x_{t-1}, x_{t-1}, x_{t-3}$. Dữ liệu trước khi đưa vào học sẽ được chuyển sang dạng chuẩn hóa trong $[0,1]$ theo công thức
\begin{equation}
	x'_{i} = \frac{x_i - min(X)}{max(X) - min (X)}
\end{equation}
với $x_i$ trong tập học.

Một số độ đo được lựa chọn để tìm ra mạng nơ ron nhân tạo có kiến trúc và trọng số phù hợp nhất như RMSE (căn quân phương của trung bình bình phương lỗi),MAE(trung bình của sai số tuyệt đối), E(chỉ số hiệu quả),$R^2$(chỉ số xác định), sMAPE(trung bình đối xứng của tỉ lệ lỗi) ... để xác định độ chính xác của dự đoán
\begin{equation}
	RMSE = \sqrt\frac{1}{n} \sum\limits_{i=1}^{n}(x_i - \hat{x_i})^{2}
\end{equation}

\begin{equation}
	MAE = \frac{1}{n} \sum\limits_{i=1}^{n}|x_i - \hat{x_i}|
\end{equation}

\begin{equation}
E = 1 - \frac{\sum(x_i-\hat{x_i})^2}{\sum(x_i-\bar{x})^2}
\end{equation}

\begin{equation}
	R^2 = \frac{\sum(x_i-\bar{x_i})(\bar{x_i}-\hat{\bar{x}})}{\sqrt{\sum((x_i-\bar{x})^2\sum((\bar{x_i}-\bar{\hat{x}})^2}}
\end{equation}

\begin{equation}
sMAPE =\frac{1}{n} \sum\frac{|x_i-\hat{x_i}|}{\frac{|x_i|+|\hat{x_i}|}{2}}*100
\end{equation}
Trong đó, độ đo $sMAPE$(Symmetric mean absolute percentage error) là độ đo chính bởi nó không chỉ dựa vào chỉ số lỗi mà còn dựa vào độ lớn dữ liệu được dự báo. Trong một số trường hợp với dữ liệu có giá trị lớn thì lỗi được tính bằng hiệu giá trị dự báo và giá trị thực tế không phản ánh được mức độ sai số bằng tỉ lệ độ lớn chênh lệch đó so với độ lớn của giá trị dự báo và giá trị thực tế.

\subsubsection{Phương pháp huấn luyện lan truyền ngược}
Với mô hình sử dụng phương pháp huấn luyện lan truyền ngược, ta sử dụng thêm kết nối truyền thẳng(skip-layer) từ mỗi điểm ở lớp đầu vào tới điểm ở lớp đầu ra. Với mô hình mạng nơ ron có 3 nút đầu vào và một nút đầu ra, số nơ ron lớp ẩn được thực nghiệm từ $2$ đến $9$. Dựa vào kết quả kiểm tra các độ đo sai số ở Bảng~\ref{nn_err}, ta nhận thấy mô hình $3:2:1$ với 2 nơ ron lớp ẩn cho kết quả các độ đo sai số nhỏ nhất. 

\begin{table}[htbp]
  \centering
  \caption{Các độ đo sai số dự đoán của mô hình mạng nơ ron sử dụng lan truyền ngược có 3 nút đầu vào}
    \begin{tabular}{cccccc}
    \\hid-neurons & rmse  & mae   & e     & R2    & sMAPE  \\
    2     & 4.831967 & 3.934214 & 1.01993 & 0.436791 & 3.09247 \\
    3     & 5.608044 & 4.277518 & 1.37387 & 0.303458 & 3.350934 \\
    4     & 5.774985 & 4.661212 & 1.456882 & 0.320368 & 3.642291 \\
    5     & 10.81326 & 8.214711 & 5.107821 & -0.21396 & 6.25575 \\
    6     & 6.632455 & 5.0617 & 1.921637 & 0.23636 & 3.9351 \\
    7     & 6.188345 & 4.542465 & 1.672906 & 0.265164 & 3.544303 \\
    8     & 5.356544 & 4.273932 & 1.253407 & 0.340559 & 3.355468 \\
    9     & 6.462559 & 4.962741 & 1.824448 & 0.213298 & 3.884248 \\
    \end{tabular}%
  \label{nn_err}%
\end{table}%

Giá trị trọng số các liên kết được mô tả trong Bảng~\ref{nn_weight} với $i$ là nút đầu vào, $h$ là nút ẩn, $o$ là nút đầu ra, $b$ là nút trực tiếp cung cấp giá trị ngưỡng.
  \begin{table}[htbp]
  \centering
  \caption{Trọng số trong mạng nơ ron 3:2:1 lan truyền ngược}
    \begin{tabular}{cccccc}
    b->h1 & i1->h1 & i2->h1 & i3->h1 &       &  \\
    5.13  & -2.01 & -9.06 & -3.44 &       &  \\ \hline
    b->h2 & i1->h2 & i2->h2 & i3->h2 &       &  \\
    0.15  & 4.35  & -4.44 & -0.52 &       &  \\ \hline
    b->o  & h1->o & h2->o & i1->o & i2->o & i3->o \\
    1.43  & -0.06 & -2.52 & 3.59  & -2.84 & -0.25 \\
    \end{tabular}%
  \label{nn_weight}%
\end{table}%
 
 Kết quả dự báo được mô tả ở Hình~\ref{3_pr_ann}
\figuremacroW{3_pr_ann}{So sánh kết quả dự đoán 23 ngày tiếp theo sử dụng mô hình ANN sử dụng giải thuật lan truyền ngược}{đường dự báo - nét đứt,đường thực tế - nét liền}{1}
Quan sát mô hình có thể thấy mô hình dự đoán có hình dạng gần giống chuỗi thời gian thực tế nhưng bị trễ một nhịp.


\subsubsection{Sử dụng giải thuật di truyền}
Thực nghiệm sử dụng mô hình mạng nơ ron áp dụng giải thuật di truyền có thêm một số tham số như
\begin{itemize}
\item tỉ lệ lai ghép $crossover-rate = 0.6$
\item tỉ lệ biến dạng $mutation-rate = 0.2$
\item Số vòng lặp (tương ứng với số thế hệ được tạo ra) $maxGen = 1500$
\item Số cá thể trong quần thế là 500
\item Tỉ lệ lỗi cho phép 0.001
\end{itemize}

Tương tự như mạng nơ ron sử dụng phương pháp lan truyền ngược, phần thực nghiệm mạng nơ ron với giải thuật di truyền cũng kiểm tra sai số đối với các mô hình có 3 nút đầu vào, 1 nút đầu ra và số nút ẩn từ 2 đến 9. Kết quả sai số thu được như Bảng~\ref{nnga_err}:

\begin{table}[htbp]
  \centering
  \caption{Các độ đo sai số dự đoán của mô hình mạng nơ ron sử dụng giải thuật di truyền có 3 nút đầu vào}
    \begin{tabular}{cccccc}
    hid\_neurons & rmse  & mae   & e     & R2    & sMAPE \\ \hline
    2     & 4.655608 & 3.794555 & 0.946837 & 0.433741 & 2.975085 \\
    3     & 6.163781 & 5.435273 & 1.659652 & 0.451339 & 4.248716 \\
    4     & 4.908285 & 4.242395 & 1.052403 & 0.44639 & 3.327021 \\
    5     & 5.331018 & 4.765645 & 1.241489 & 0.450103 & 3.733707 \\
    6     & 4.814318 & 4.005334 & 1.012493 & 0.438789 & 3.140415 \\
    7     & 7.126121 & 6.270289 & 2.218344 & 0.43476 & 4.872308 \\
    8     & 7.233547 & 6.353208 & 2.285731 & 0.442229 & 4.935695 \\
    9     & 5.457268 & 4.710235 & 1.300988 & 0.439929 & 3.684482 \\
    \end{tabular}%
  \label{nnga_err}%
\end{table}%

Quan sát Bảng~\ref{nnga_err} thì với các chỉ số sai số nhỏ nhất, mạng 3:2:1 vẫn thích hợp hơn cả. Sai số của nó so với mạng noron sử dụng giải thuật lan truyền ngược nhỏ hơn nhưng không đáng kể.

Ta có kết quả dự đoán Hình~\ref{3_pr_ga}.
Độ fitness tốt nhất của quần thể : $0.00298121$. Nhiễm sắc thể tốt nhất tìm được tương ứng với tập trọng số tốt nhất là $-5.43/0.62/-0.53/0.90/10.40/-0.91/-0.53/-8.34/-4.94/2.56/0.70/$

\figuremacroW{3_pr_ga}{So sánh kết quả dự đoán 23 ngày tiếp theo sử dụng mô hình mạng nơ ron sử dụng GA}{}{1}
Mô hình chuỗi thời gian dự đoán bởi mạng nơ ron mới xây dựng với đầu vào là tập học(160 ngày) có hình dáng khá giống với chuỗi thời gian thực tế(Hình~\ref{3_ga_old}). Ta hoàn toàn có thể tin tưởng được.
\figuremacroW{3_ga_old}{So sánh kết quả mô hình ANN-GA dự đoán trên tập học}{(Đường dự đoán - nét liền)}{1}

\section{So sánh kết quả}
So sánh độ đo lỗi của đại diện 2 phương pháp: ARIMA và mạng nơ ron nhân tạo ta có được kết quả như Bảng~\ref{ss}. Kết quả so sánh các mô hình đã thử nghiệm cho thấy 2 mạng nơ ron sử dụng 2 phương pháp học khác nhau cho kết quả gần bằng nhau và tốt hơn so với mô hình ARIMA. Trung bình tỉ lệ lỗi $w_t$ so với giá trị của chuỗi thời gian $x_t$ của mô hình mạng nơ ron là khoảng 2.975\%-3.092\%.

% Table generated by Excel2LaTeX from sheet 'New Microsoft Excel Worksheet ('
\begin{table}[htbp]
  \centering
  \caption{So sánh độ đo lỗi 3 mô hình}
    \begin{tabular}{crcccc}
          & \multicolumn{1}{c}{rmse} & mae   & e     & R2    & sMAPE \\ \hline
    GA-ANN(3:2:1) & \multicolumn{1}{c}{4.655608} & 3.794555 & 0.946837 & 0.433741 & 2.975085 \\
    ANN(3:2:1) lan truyền ngược & 4.831967 & 3.934214 & 1.01993 & 0.436791 & 3.09247 \\
    ARIMA & \multicolumn{1}{c}{8.21929} & 6.573086 & 2.95115 & 0.506937 & 5.349483 \\
    \end{tabular}%
  \label{ss}%
\end{table}%

