\begin{abstractsvn}       Bài toán phân tích và dự báo chuỗi thời gian luôn là bài toán thu hút sự quan tâm rất nhiều nhà khoa học, kinh tế học ngay từ những thập niên 70 của thế kỉ trước. Bài toán giúp dự báo những giá trị tương lai dựa vào những giá trị, trạng thái hiện tại và quá khứ. Ban đầu, các nhà khoa học sử dụng những phương thống kê từ đơn giản đến phức tạp để tìm ra mô hình xấp xỉ ẩn đằng sau chuỗi dữ liệu, sao đó sử dụng mô hình đó dự báo cho tương lại. Ngay nay, bài toán dự báo chuỗi thời gian còn sử dụng các kĩ thuật học máy tiên tiến được biến đổi từ các lớp bài toán khác đã tạo ra hiệu quả và sự khác biệt mới.

Dựa trên khảo sát những kiểu dữ liệu và kĩ thuật áp dụng hiện có, khóa luận đề xuất sử dụng phương pháp truyền thống dựa trên mô hình thông kê ARIMA (mô hình tự hồi quy – trung bình trượt)\cite{bow79} theo quy trình lặp của Box-Jenkins và mô hình mạng nơ ron nhân tạo sử dụng phương pháp huấn luyện lan truyền ngược và giải thuật di truyền để giải bài toán dự báo lưu lượng kênh truyền của mạng di động Beeline với dữ liệu từ các bản ghi chi tiết cuộc gọi, dịch vụ...Kết quả cho thấy, mô hình mạng nơ ron nhân tạo với 3 nút đầu vào, 2 nút ẩn và 1 nút đầu ra cho kết quả tốt hơn với các độ đo sai số nhỏ hơn của ARIMA. Với mạng nơ ron, trung bình tỉ lệ lỗi trên giá trị $sMAPE = 2.975\% - 3.092\%$ trong khi của mô hình $ARIMA(0,1,2)(2,0,2)_7$ là $5.34\%$. 

\end{abstractsvn}
%\end{abstractlongs}

% ---------------------------------------------------------------------- 
