\chapter{Tính chất của chuỗi thời gian}
% the code below specifies where the figures are stored
\ifpdf
    \graphicspath{{01_timeseries/figures/PNG/}{01_timeseries/figures/PDF/}{01_timeseries/figures/}}
\else
    \graphicspath{{01_timeseries/figures/EPS/}{01_timeseries/figures/}}
\fi

\section{Chuỗi thời gian}
Trong quá trình theo dõi sự thay đổi hiện tượng, ta đòi hỏi có một số lượng lớn các quan sát cho các đại lượng thích hợp để nghiên cứu các mối quan hệ giữa các đại lượng này. Các quan sát này có thể được tiến hành đều đặn qua các thời kỳ chẳng hạn theo từng ngày, từng tuần, từng tháng, từng quý hoặc từng năm. Dãy các quan sát này gọi là chuỗi thời gian. Như vậy chuỗi thời gian là tập hợp các quan sát được ghi nhận tại các thời điểm t với $t \in T$ . Chuỗi thời gian được gọi là rời rạc nếu T là tập các điểm rời rạc (Thí dụ các quan sát được thực hiện cách nhau một khoảng thời gian đều đặn như doanh thu cước phí điện thoại hàng tháng của một trạm bưu điện từ tháng 1 năm 1990 đến tháng 12 năm 2000). Ngược lại nếu T là một khoảng thì chuỗi thời gian là liên tục. Biểu đồ ghi nhịp tim của một bệnh nhân trong 3 giờ là một ví dụ minh họa cho chuỗi thời gian liên tục với T =  [0,3].

Hầu hết các thủ tục thống kê đều dùng những số liệu xuất phát từ một loạt các quan sát độc lập tập hợp thành một mẫu quan sát và ký hiệu là $X =\{x_1,x_2,...,x_n\}$(ứng với $n$ quan sát). Trong phân tích thống kê cổ điển người ta không quan tâm tới thứ tự thời gian quan sát diễn ra trong mẫu. Tuy nhiên với mẫu quan sát dưới dạng chuỗi thời gian thì phương pháp này lại không phù hợp vì nó sẽ làm mất đi tính tuần tự của dữ liệu.

Để có thể mô tả tính chất quan trọng này của dữ liệu chuỗi thời gian, giúp tạo điều kiện lựa chọn và đánh giá áp dụng các mô hình phân tích chính xác hơn, chúng ta coi một chuỗi thời gian là một tập hợp các biến ngẫu nhiên được đánh chỉ số theo thứ tự thời gian. Hay nói cách khác, một \textbf{chuỗi thời gian}\citep{ross} là một dãy các giá trị quan sát $X =\{x_1,x_2,...,x_n\}$ được xếp thứ tự theo diễn biến thời gian, $x_1$ là giá trị quan sát tại thời điểm đầu tiên, $x_2$ là giá trị tại thời điểm quan sát thứ hai, … và $x_n$  là giá trị tại thời điểm quan sát thứ n (cũng là thời điểm cuối cùng). Chuỗi thời gian sẽ được biểu diễn bởi một quá trình thông kê ngẫu nhiên $\{x_t\}$ với $t$ có thể liên tiếp $t = 0, \pm1,\pm2,...$ hoặc đơn giản là một tập các số nguyên có thứ tự. Trên thực tế, mức tiêu thụ điện theo từng tháng của một hộ gia đình, lượng hành khách hàng ngày trên chuyến tàu Bắc Nam...tất cả đều là thể hiện cụ thể của các chuỗi thời gian. Trong các thí dụ trên, thứ tự thời gian quan sát thực sự đóng vai trò quan trọng, vì thế hầu hết các kỹ thuật thống kê cổ điển ít có tác dụng và do đó cần phải đề xuất những kỹ thuật tính toán mới để bộc lộ được các nét đặc thù của chuỗi thời gian. 

Vậy chuỗi thời gian là một chuỗi các giá trị ngẫu nhiên của một đại lượng nào đó được ghi nhận tuần tự theo thời gian.

Tất cả các kỹ thuật dự báo truyền thống theo chuỗi thời gian dựa trên giả định là có một mẫu hình cơ bản tiềm ẩn trong các số liệu nghiên cứu cùng với các yếu tố ngẫu nhiên ảnh hưởng lên hệ thống đang xét. Công việc của phân tích chuỗi thời gian là nghiên cứu kỹ thuật để tách mẫu hình cơ bản này và sử dụng nó như là cơ sở để sản sinh ra dữ liệu dự báo cho tương lai.

\section{Tính chất đặc trưng của chuỗi thời gian}
Các tính chất đặc trưng chính \citep{cits} của chuỗi thời gian là \textbf{tính dừng, tính tuyến tính, tính xu hướng và tính chu kì thời vụ}. Mặc dù, một chuỗi thời gian có thể mang một hoặc nhiều các đặc trưng trên nhưng khi biểu diễn, phân tích hay dự báo giá trị thì mỗi đặc trưng đều có phương pháp đánh giá và xử lý riêng rẽ. 

\subsection{Tính dừng}
Trong khái niệm xác suất thì một chuỗi có tính chất dừng khi nó được mô tả bởi một quá trình ngẫu nhiên mà quá trình này nằm trong trạng thái cân bằng xác suất(statistical equilibrium) tức là phân phối xác suất chung của $x(t)$ và $x(t+h)$ chỉ phụ thuộc vào $h$ mà không phụ thuộc vào $t$~\citep{cits}. Do đó, hoàn toàn có thể xây dựng mô hình ngẫu nhiên có tính chất dừng cho một chuỗi thời gian luôn giữ trạng thái cân bằng quanh giá trị trung bình.

Phần~\ref{sec:sta} sẽ thảo luận rõ hơn về tính dừng của một chuỗi thời gian vì nó là một trong những tính chất quan trọng được nhắc đến trong hầu hết các mô hình thống kê về chuỗi thời gian. Nhưng có thể quan sát đánh giá một cách tương đối tính dừng của một của một chuỗi thời gian. Một chuỗi dừng thường có đồ thị nhìn khá phẳng, không có xu hướng, biến thiên đều. (Hình~\ref{1_Stationarycomparison2}).
\figuremacroW{1_Stationarycomparison2}{Chuỗi dừng và không có tính chất dừng}{}{0.5}

\subsection{Tính tuyến tính}
\begin{defi}
	Một chuỗi thời gian được gọi là tuyến tính\citep{afts3} nếu nó có thể được biểu diễn bởi công thức:
	\begin{equation}
		X_t = \mu+\sum\limits_{i = -\infty}^{\infty}\psi_iZ_{t-i}
	\end{equation}
	với $\mu$ là giá trị trung bình của $\{x_t\}$, $\{ \psi_i\}$ là tập tham số thỏa mãn $\sum\limits_{i = -\infty}^{\infty}|\psi_i| < \infty$. Và $|Z_t|$ là chuỗi ngẫu nhiên có giá trị trung bình bằng $0$ và phương sai $\sigma^2$
\end{defi}
Do đó, nếu một chuỗi thời gian là tuyến tính thì nó có thể được biểu diễn bởi một hàm phụ thuộc tuyến tính các giá trị hiện tại và quá khứ.

\subsection{Tính xu hướng} 
Tính xu hướng của chuỗi dữ liệu thể hiện qua hiện tượng tăng hoặc giảm giá trị trung bình của các đoạn con một cách liên tục trong phạm vi cục bộ hoặc cả chuỗi. Để xác định những thành phần có tính xu hướng trong chuỗi thời gian, dữ liệu quan sát được thường được ước lượng xấp xỉ vào trong các mô hình hồi quy. Một số mô hình thông thường hay được sử dụng như :
\begin{align}
	x_t &= \alpha{t}+ \beta + \varepsilon_t \\
	x_t &= exp(\alpha{t}+ \beta + \varepsilon_t) \\
	x_t &= \alpha{t} + \beta{t}^{2} + \gamma + \varepsilon_t
\end{align}
\figuremacroW{1_trend}{Xu hướng giảm dần}{}{0.5}
\subsection{Tính chu kỳ thời vụ(seasonality)}
Một số chuỗi thời gian xuất hiện các yếu tố dao động theo chu kỳ. Đó chính là đặc trưng mang tính thời vụ rất hay thường gặp trong dữ liệu quan sát thực tế, nhất là trong các dữ liệu về tài chính, kinh tế. Nó thường xuất hiện các đoạn dao động lặp lại giống nhau theo từng giờ, từng ngày, từng tháng, từng quý...Ví dụ như lượng dòng chảy đến hồ chứa Trị An từ năm 1959 đến 1985 (Hình~\ref{1_sea})
\figuremacroW{1_sea}{Lượng dòng chảy đến hồ chứa Trị An từ năm 1959 đến 1985}{}{1}

\section{Phân loại chuỗi thời gian}
Phụ thuộc vào tính chất của dữ liệu mà chuỗi thời gian có thể chia theo một số tiêu chí\citep{cits} sau:
\begin{itemize}
	\item có tính chất dừng và không có tính chất dừng
	\item có tính chu kỳ thời vụ và không có tính chu kỳ thời vụ
	\item tuyến tính và không tuyến tính
	\item đơn chiều và đa chiều
	\item hỗn loạn
\end{itemize}
Chuỗi thời gian \textbf{đơn chiều} là chuỗi chỉ ghi lại  kết quả quan sát theo thời gian của một dạng dữ liệu. Ví dụ như trong một phiên chứng khoán, có rất nhiều giá trị được quan sát như khối lượng giao dịch, giá đóng, giá mở... nhưng chuỗi thời gian đơn chiều chỉ quan tâm đến một trong những giá trị đó. \textit{Khóa luận sẽ tập trung vào phân tích xử lý đối với chuỗi dữ liệu dạng này}. 

Ngược lại, chuỗi thời gian \textbf{đa chiều} ghi lại kết quả quan sát của hai hay nhiều quá trình đồng thời.Ví dụ, ta có thể ghi lại đồng thời tất cả các giá trị trong phiên chứng khoán ở ví dụ trên. Trong chuỗi đa chiều, dữ liệu của một chuỗi đơn không chỉ phụ thuộc lẫn nhau trong nội bộ chuỗi mà còn phụ thuộc hai chiều với dữ liệu của các chuỗi đơn khác. 

\section{Đo độ phụ thuộc: Hàm tự tương quan và tương quan chéo}
Trong trường hợp $x_t$ là biến ngẫu nhiên liên tục có hàm phân phối xác suất $F_t(x)=P(x_t\leq x)$ thì hàm mật độ xác suất được tính theo công thức:
	\begin{equation}
		f_t(x)  = \frac{\partial F_t(x)}{\partial x}
	\end{equation}
	\begin{defi} Hàm tính kỳ vọng (giá trị trung bình)
		\begin{equation}
			\mu_{xt} = E(x_t) = \int_{-\infty}^\infty \! xf_x(x) \, dx
		\end{equation}
	\end{defi}
	
	\begin{defi}
		Hàm \textbf{tự hiệp phương sai} được xác định bởi công thức:
		\begin{equation}
			\gamma_x(s,t) = cov(x_s,x_t) = E[(x_s-\mu_s)(x_t-\mu_t)]
		\end{equation}	
		với mọi thời điểm $s$, $t$
	\end{defi}
	Ta luôn có $\gamma_x(s,t) = \gamma_x(t,s)$ với mọi $s,t$. Hàm tự hiệp phương sai giúp đánh giá độ phụ thuộc tuyến tính giữa hai thời điểm khác nhau trong chuỗi thời gian thực. Trong trường hợp $s=t$ thì ta có giá trị hàm tự hiệp phương sai chính bằng phương sai của $x_s$ hay $\gamma(s,s) = E[(x_s-\mu_s)^2] = var(t)$ 

\begin{defi}
	Hàm tự tương quan (ACF)\citep{tsa3} được xác định bởi:
	\begin{equation}
		\rho(s,t) = \frac{\gamma(s,t)}{\sqrt{\gamma(s,s)\gamma(t,t)}}
	\end{equation}
\end{defi}
Hàm ACF cho biết khả năng dự đoán giá trị $x_t$ mà chỉ dựa vào giá trị của $x_s$. Dựa vào bất đẳng thức Cosi\footnote{Bất đẳng thức Cosi: $|\gamma(s,t)|^{2} \leq \gamma(s,s)\gamma(t,t)$}, ta dễ dàng có được $-1\leq \rho(s,t)\leq 1$. Nếu ta có thể dự đoán giá trị $x_{s}$ hoàn toàn chỉ dựa vào $x_t$ thông qua hàm tuyến tính $x_t = \beta_0 + \beta_1x_s$ thì sự tương quan sẽ là $+1$ nếu $\beta_1 > 0$ và là $-1$ nếu $\beta_1<0$. Vậy ta có một công thức đơn giản để đánh giá khả năng dự báo giá trị tại thời điểm $t$ dựa vào giá trị tại thời điểm $s$ trong chuỗi thời gian.

Ngoài ra, chúng ta còn có thể đo khả năng có thể dự đoán chuỗi $y_t$ từ chuỗi $x_t$ thông qua \textit{hàm tự tương quan chéo}. Giả sử cả hai chuỗi đều có phương sai xác định, ta có các khái niệm sau:
\begin{defi}
Hàm \textbf{tự hiệp phương sai chéo} của hai chuỗi $x_t$ và $y_t$ được cho bởi công thức
\begin{equation}
		\gamma_{xy}(s,t) = cov(x_s,y_t) = E[(x_s - \mu_{xs})(y_t - \mu_{yt} )]
\end{equation}
\end{defi}
\begin{defi}
\textbf{Hàm tự tương quan chéo(CCF)}\citep{tsa3} được cho bởi:
	\begin{equation}
		\rho_{xy}(s,t) = \frac{\gamma_{xy}(s,t)}{\sqrt{(\gamma_s(s,s))(\gamma_y(t,t))}}
	\end{equation}
\end{defi}

\section{Chuỗi dừng} \label{sec:sta}
Chuỗi dừng và không dừng là một trong những khái niệm quan trọng cốt lõi trong việc phân tích chuỗi thời gian.
\subsection{Chuỗi dừng và tính chất}
\begin{defi}
	Chuỗi $x_t$ được gọi là \textbf{dừng theo nghĩa chặt}\citep{tsa3} nếu như tại mọi thời điểm $t_1,t_2,\dots,t_k$ thì ta luôn có $(x_{t_1},x_{t_2},\dots,x_{t_k})$ có cùng phân phối xác suất với $(x_{t_1+h},x_{t_2+h},\dots,x_{t_k+h})$ hay
	\begin{equation}
		P(x_{t_1} \leq c_1,x_{t_2} \leq c_2,\dots,x_{t_k} \leq c_k) = P(x_{t_1+h} \leq c_1,x_{t_2+h} \leq c_2,\dots,x_{t_k+h} \leq c_k)
	\end{equation}
	với $k > 0 , h > 0$ và mọi số $c_k \geq 0$
\end{defi}
Khái niệm chuỗi dừng theo nghĩa chặt khá nghiêm ngặt cho hầu hết các chuỗi dữ liệu. Nó yêu cầu các phân phối đồng thời của $x_t$ bất biến qua phép biến đổi dịch chuyển thời gian. Ví dụ với $k=2$ ta có 
\begin{equation}
 P(x_s \leq c_1,x_t \leq c_2) = P(x_{s+h} \leq c_1,x_{t+h} \leq c_2)
\end{equation} với $s,t,h$ bất kì.Khi đó nếu tồn tại giá trị phương sai của chuỗi thì hàm tự hiệp phương sai của $x_t$ thỏa mãn 
\begin{equation}
	\gamma(s,t) = \gamma(s+h,t+h)
\end{equation}
Khi đó hàm tự hiệp phương sai chỉ còn phụ thuộc vào khoảng thời gian bước nhảy $h$ bất kì chứ không phải phụ thuộc vào thời điểm $s$ hoặc $t$. Do đó, đôi khi người ta chỉ đòi hỏi một vài điều kiện lỏng hơn trên một trong hai momment của chuỗi. 
 \begin{defi}
  Chuỗi $x_t$, có phương sai xác định, được gọi là \textbf{dừng theo nghĩa rộng} \citep{tsa3} nếu
  \begin{enumerate}
  \item Hàm kỳ vọng $\mu_t$ là hằng số và không phụ thuộc vào thời gian $t$
  \item Hàm hiệp tự phương sai, $\gamma(s,t)$, phụ thuộc vào s và t chỉ thông qua hiệu $|s-t|$ 
  \end{enumerate}
 \end{defi}
 Đặt $s=t+h$ với $h$ là khoảng cách thời gian thì theo tiêu chí thứ 2 ở trên, ta có
 \begin{equation}
 \gamma(t+h,t) = cov(x_{t+h},x_t) = cov(x_h,x_0) = \gamma(h,0)
 \end{equation}
 Khi đó hàm tự hiệp phương sai sẽ không còn phụ thuộc vào $t$ mà phụ thuộc vào độ trễ.
 
 Trong giới hạn khóa luận, ta dùng thuật ngữ \textbf{chuỗi dừng} để ám chỉ định nghĩa \textbf{"chuỗi dừng theo nghĩa rộng"}

Áp dụng một số hàm số cho chuỗi dừng

\begin{defi}
Hàm tự hiệp phương sai cho chuỗi dừng:
	\begin{equation}
		\gamma(h) = cov(x_{t+h}) = E[(x_{t+h}-\mu)(x_t-\mu)]
	\end{equation}
\end{defi}

\begin{defi}
Hàm tự tương quan ACF cho chuỗi dừng:
\begin{equation}
	\rho(h) = \frac{\gamma(t+h,t)}{\sqrt{\gamma(t+h,t+h)\gamma(t,t)}}=\frac{\gamma(h)}{\gamma(0)}
\end{equation}
\end{defi}
Dựa vào bất đẳng thức Cosi ta cũng có $|\rho(h)| \leq 1$. Do đó, có thể so sánh giá trị với $-1$ và $1$ để đánh giá mối quan hệ của hai thời điểm trong hàm tự tương quan.

Một số tính chất của hàm ACF cho chuỗi dừng:
\begin{enumerate}
	\item Với $h=0$ ta có $\gamma(0) = E[(x_t-\mu)^2] = var(x_t)$ 
	\item $\gamma(h)=\gamma(-h)$
\end{enumerate}
\subsection{Kiểm tra chuỗi dừng}\label{1.2.2}
Trong một số nghiên cứu, một số dữ liệu như tỉ lệ lợi tức, tỉ giá tiền đổi hay giá cả thị trường... có thể là dữ liệu không có tính chất dừng. Chúng ta gọi chúng là những chuỗi không dừng nghiệm đơn vị.
Dựa vào khái niệm thì một chuỗi dừng sẽ thỏa mãn 2 điều kiện là hàm tính kỳ vọng luôn cho giá trị hằng số duy nhất và hàm tự hiệp phương sai, $\gamma(s,t)$ chỉ phụ thuộc vào $h=|s-t|$. Do đó ta có một số phương pháp để kiểm tra một chuỗi có phải là chuỗi dừng không?
\subsubsection{Đồ thị hóa giá trị trung bình và ACF}
Khi mô hình hóa chuỗi thời gian, ta nhận thấy chuỗi có xu hướng tăng dần, giảm dần hoặc giá trị trung bình biến đổi thì chuỗi đó chắc chắn không phải là chuỗi dừng theo tính chất thứ nhất trong định nghĩa.
Ngoài ra, khi quan sát mô hình trực quan hàm tự tương quan ACF, nếu giá trị đạt cao nhất ở thời điểm đầu tiên, sau đó giảm chậm dần thì đó là một chuỗi không có tính dừng \citep{bow79}.
\subsubsection{Kiểm tra nghiệm đơn vị (unit-root)}
\begin{defi}[Nghiệm đơn vị]\label{unit_root}
Giả sử một quá trình ngẫu nhiên có thể được viết dưới dạng:
\begin{equation}
	y_t = a_1y_{t-1}+a_2y_{t-2}+...+a_py_{t-p}+\epsilon_t
\end{equation}
	với $a_1,a_2,\dots$ là các tham số; $\epsilon_t$ là quá trình ngẫu nhiên có xấp xỉ bình phương bằng 0 và phương sai không đổi bằng $\sigma^{2}$. Không mất tính tổng quát coi $y_0 = 0$.
	Khi đó nếu $m=1$ là nghiệm của phương trình 
	\begin{equation}
		m^p - m^{p-1}a_1 - m^{p-2}a_2 - ... - a_p = 0 
	\end{equation}
	thì quá trình ngẫu nhiên này có một nghiệm đơn vị $m$
\end{defi}
Khi xảy ra nghiệm đơn vị thì đây sẽ không phải là chuỗi dừng nữa vì mô hình sẽ có thể có xu hướng, giá trị trung bình sẽ biến đổi hay phương sai phụ thuộc thời gian. Do đó, nó không thỏa mãn điều kiện chuỗi dừng. Ví dụ, giả sử đối với một chuỗi $x_t$ đơn giản
\begin{equation}
	x_t = \phi_1x_{t-1}+w_1
\end{equation}
với tham số $\phi_1$ và thành phần nhiễu $w_1$. Trong trường hợp này, nghiệm đơn vị sẽ xảy ra khi $\phi_1=1$.
Khi đó ta sẽ có 
	\begin{align}
		x_t&= x_{t_1}+w_1 \\
		x_t&=x_0+\sum\limits_{j=1}^{t}w_j\\
		var(x_t) &= \sum\limits_{j=1}^{t}\sigma^2 = t\sigma^2
	\end{align}
	Phương sai phụ thuộc vào thời gian $t$. Do đó, hàm tự hiệp phương sai cũng phụ thuộc vào $t$. Đây sẽ không phải là chuỗi dừng.
Một số phương pháp kiểm tra nghiệm đơn vị như Dickey–Fuller \citep{df79} áp dụng cho mô hình tự hồi quy, mở rộng Augmented Dickey–Fuller \citep{ell96} cho mô hình ARMA hay phương pháp Phillips–Perron \citep{ff88} sẽ được nhắc đến sau.
Một trong những phương pháp phổ biến để chuyển một chuỗi có tính chất dừng thành chuỗi dừng là phương pháp sai phân
\begin{equation}
	c_t = x_{t} - x_{t-1}
\end{equation}

\section{Bài toán dự báo chuỗi thời gian}
\subsection{Bài toán dự báo}
Trong nền kinh tế thị trường, công tác dự báo là vô cùng quan trọng bởi lẽ nó cung cấp các thông tin cần thiết nhằm phát hiện và bố trí sử dụng các nguồn lực trong tương lai một cách có căn cứ thực tế. Trong môi trường phong phú và có tính cạnh tranh cao, thông tin là vô cùng quan trọng. Người nắm được thông tin trước là người đi đầu, chớp được thời cơ. Đó là một phần cực kì quan trọng ảnh hướng đến lợi nhuận, đường lối phát triển của doanh nghiệp. Với những thông tin mà dự báo đưa ra cho phép các nhà hoạch định chính sách có những quyết định về đầu tư, các quyết định về sản xuất, về tiết kiệm và tiêu dùng, các chính sách tài chính, chính sách kinh tế vĩ mô. Dự báo không chỉ tạo cơ sở khoa học cho việc hoạch định chính sách, cho việc xây dựng chiến lược phát triển, cho các quy hoạch tổng thể mà còn cho phép xem xét khả năng thực hiện kế hoạch và hiệu chỉnh kế hoạch.

Trong quản lý vi mô, dự báo là hoạt động gắn liền với công tác hoạch định và chỉ đạo thực hiện chiến lược kinh doanh của doanh nghiệp. Các doanh nghiệp không thể không tổ chức thực hiện tốt công tác dự báo nếu họ muốn đứng vững trong kinh doanh.

Chức năng đầu tiên của  quản lý trong doanh nghiệp là xác định mục tiêu của doanh nghiệp trong dài hạn và ngắn hạn. Doanh nghiệp phải lập kế hoạch để thực hiện những mục tiêu đó, tổ chức tốt các nguồn nhân lực và vật tư để thực hiện kế hoạch, điều chỉnh kế hoạch cũng như kiểm soát các hoạt động để tin chắc rằng tất cả diễn ra theo đúng kế hoạch. Phân tích kinh tế và dự báo được tiến hành trong tất cả các bước của quản lý doanh nghiệp, nhưng trước hết là trong việc xác định mục tiêu và hoạch định các kế hoạch dài hạn và ngắn hạn.

Trong việc xác định mục tiêu, mỗi doanh nghiệp phải quyết định hàng hóa và dịch vụ nào sẽ được sản xuất và bán ra, mức giá sản phẩm và dịch vụ, vùng tiêu thụ, thị trường tiềm năng về sản phẩm đó. Thị phần mà doanh nghiệp thực tế có thể hi vọng chiếm được, hiệu suất vốn doanh nghiệp có thể kỳ vọng...Những mục tiêu như vậy chỉ có thể trở thành hiện thực nếu doanh nghiệp đã phân tích các xu thế của nền kinh tế, đã dự báo về nhu cầu sản phẩm của mình cả trong dài hạn và ngắn hạn, chi phí các nhân tố sản xuất...Như vậy các dự báo về thị trường, giá cả, tiến bộ khoa học và công nghệ, nguồn nhân lực, sự thay đổi của các nguồn đầu tư vào, đối thủ cạnh tranh,... có tầm quan trọng sống còn đối với doanh nghiệp. Ngoài ra dự báo cung cấp những thông tin cho phép phối hợp hành động giữa các bộ phận trong doanh nghiệp.

Hiện nay có rất nhiều phương pháp dự báo khác nhau về nguồn thông tin được sử dụng, về cơ chế xây dựng dự báo, về độ tin cậy độ xác thực của dự báo. Tuy nhiên, có một số mô hình thông dụng hay được sử dụng \citep{efin} như:
\begin{description}
\item[Mô hình kinh tế lượng:] Là phương pháp dựa trên lý thuyết kinh tế lượng để lượng hoá các quá trình kinh tế xã hội thông qua phương pháp thống kê. ý tưởng chính của phương pháp là mô tả mối quan hệ giữa các đại lượng kinh tế bằng một phương trình hoặc hệ phương trình đồng thời. Với các số liệu quá khứ, tham số của mô hình này được ước lượng bằng phương pháp thông kê. Sử dụng mô hình đã ước lượng này để dự báo bằng kỹ thuật ngoại suy hoặc mô phỏng.
\item [Mô hình I/O:] Mô hình I/O là mô hình dựa trên ý tưởng là mối liên hệ liên ngành trong bản đầu vào - đầu ra(Input – Output table) diễn tả mối quan hệ của quá trình sản xuất giữa các yếu tố đầu vào, chi phí trung gian và đầu ra của sản xuất.
\item[Mô hình tối ưu hoá:] Điển hình của mô hình này là bài toán quy hoạch tối ưu, bố trí một nguồn lực nhằm tối ưu hoá một mục tiêu nào đó.
\item[Mô hình chuỗi thời gian:] Phương pháp dự báo này được tiến hành trên cơ sở giả định rằng quy luật đã phát hiện trong quá khứ và hiện tại được duy trì sang tương lai trong phạm vi tầm xa dự báo. Các quy luật này được xác định nhờ phân tích chuỗi thời gian và được sử dụng để suy diễn tương lai.
\item [Mô hình nhân tố:] Phân tích tương quan giữa các chỉ tiêu(nhân tố) với nhau và lượng hoá các mối quan hệ này. Việc lượng hoá được thực hiện nhờ phương pháp phân tích hồi quy và dự báo chỉ tiêu kết quả trên cơ sở sự thay đổi của các chỉ tiêu nguyên nhân hay các chỉ tiêu giải thích.
\end{description}
Khóa luận sẽ tập trung nghiên cứu về việc sử dụng mô hình chuỗi thời gian cho bài toán dự báo nhu cầu đang cực kì cấp thiết hiện nay.
\subsection{Bài toán phân tích chuỗi thời gian và dự báo}
\subsubsection{Mô hình thống kê truyền thống}
Như đã nói ở trên, tất cả các kỹ thuật truyền thống dự báo theo chuỗi thời gian dựa trên giả định là có một mẫu hình cơ bản tiềm ẩn trong các số liệu nghiên cứu cùng với các yếu tố ngẫu nhiên ảnh hưởng lên hệ thống đang xét. Công việc của phân tích chuỗi thời gian là nghiên cứu kỹ thuật để tách mẫu hình cơ bản này và sử dụng nó như là cơ sở để sản sinh ra dữ liệu dự báo cho tương lai.

Để làm được điều đó, trước hết ta giả thiết có một mô hình xác suất để biểu diễn dãy số liệu. Sau cùng hy vọng chọn ra một mô hình gần với dãy số liệu dựa vào các độ đo, chúng ta tiến hành ước lượng các tham số của mô hình, kiểm tra lại xem mô hình được sử dụng có phù hợp không.

Dự báo dựa trên mô hình chuỗi thời gian là ước lượng các giá trị tương lai $x_{t+h}$ của một biến ngẫu nhiên dựa trên các giá trị quan sát trong quá khứ của nó $x_1,x_2,...,x_t$ . Giá trị dự báo của $x_{t+h}$ thường được ký hiệu là $\hat{x_t}(h)$.


Có 2 hướng tiếp cận chính \citep{cits} khi phân tích chuỗi dữ liệu theo phương pháp truyền thống:
\begin{description}
  \item[Miền thời gian] chủ yếu dựa vào việc sử dụng hàm hiệp phương sai của chuỗi thời gian
  \item[Miền tần số] dựa vào phân tích hàm mật độ phổ và Fourier
\end{description}
Cả hai hướng tiếp cận đều có rất nhiều ứng dụng. Một trong các phương pháp phổ biến theo hướng tiếp cận miền thời gian do George Box và Gwin Jenkins (1970)\citep{ross} đề xuất có tên là \textbf{Quá trình tự hồi quy trung bình trượt ARMA}(ARMA là AutoRegressive Moving Average. Nó là một quy trình lặp xử lý mô hình chuỗi dừng tuyến tính. Tuy nhiên, nó cũng có thể được mở rộng thành quy trình ARIMA( quy trình tự hồi quy trung bình trượt tích hợp)để xử lý với chuỗi không dừng. Trong quy trình ARIMA, dữ liệu đầu tiên sẽ được sai phân đến khi chuỗi mới biến đổi có tính dừng. Sau đó, ta áp dụng ARMA cho chuỗi mới biến đổi đó.

Ngoài ra, chất lượng của dự báo phụ thuộc vào nhiều yếu tố. Trước hết nó phụ thuộc vào xu hướng phát triển của chuỗi thời gian. Nếu chuỗi thời gian là hàm "đều đặn" theo thời gian thì càng dễ xác định giá trị dự báo. Thí dụ nếu tiến trình phát triển kinh tế không có những biến động đặc biệt thì càng dễ dàng đánh giá tổng sản phẩm quốc nội (GDP) cho những năm tiếp theo. Cho đến nay, các phương pháp dự báo dựa trên chuỗi thời gian chưa cho phép đánh giá ước lượng được các giá trị đột biến. Chất lượng của dự báo dựa vào chuỗi thời gian còn phụ thuộc vào độ xa gần của thời gian. Các giá trị dự báo càng gần hiện tại thì độ chính xác càng cao. Như vậy việc ước lượng GDP cho năm sẽ chính xác hơn việc ước lượng GDP cho 10 năm sau.
\subsubsection{Mô hình học máy}
Khi dự báo chuỗi thời gian sử dụng những kĩ thuật truyền thống như ARIMA, làm trơn, hồi quy hay phân tích xu hướng...ta phải quan tâm đến rất nhiều yếu tố tính chất ảnh hưởng đến việc xác định mô hình xấp xỉ như tính dừng, tuyến tính, chu kì...Sự phát hiện ra những thành phần mang các yếu tố trên là cực kì quan trọng, bởi mỗi yếu tố sẽ được đánh giá và định hướng ước lượng mô hình khác nhau. Mặt khác, việc phát hiện không phải lúc nào cũng dễ dàng, nếu đánh giá không đúng sẽ dẫn tới việc ước lượng mô hình với sai số khá lớn. 

Các kĩ thuật khai phá dữ liệu mới ra đời đã cố gắng giải quyết hoàn toàn vấn đề trên. Một số phương pháp cơ bản như sử dụng mạng nơ ron, hướng tiếp cận logic mờ, tính toán tiến hóa hay có một số kĩ thuật mới hơn như svm, phân lớp mờ...Vai trò của các đặc trưng tính chất trong chuỗi thời gian được giảm đi đáng kể. Một số kĩ thuật còn không quan tâm đến các tính chất đó. Chương 2 sẽ giới thiệu mô hình \textit{mạng nơ ron nhân tạo} để chứng minh điều đó. Hiệu quả của chúng so với phương pháp truyền thống ra sao sẽ được nhắc tới sâu hơn trong khóa luận.
