\chapter{Kết luận và hướng nghiên cứu}
\ifpdf
    \graphicspath{{4_conclusion/figures/PNG/}{4_conclusion/figures/PDF/}{4_conclusion/figures/}}
\else
    \graphicspath{{4_conclusion/figures/EPS/}{4_conclusion/figures/}}
\fi

\section{Kết quả}
Qua quá trình tìm hiểu bài toán dự báo với việc sử dụng mô hình thông kê ARIMA và mô hình mạng nơ ron nhân tạo (sử dụng giải thuật di truyền và lan truyền ngược), khóa luận đạt được một số kết quả sau đây:
\begin{itemize}
\item Trình bày mô hình thống kê ARIMA và các mô hình con, sử dụng thủ tục Box-Jenkins áp dụng bài tóan dự báo chuỗi thời gian đơn chiều tuyến tính.
\item Trình bày mô hình mạng nơ ron nhân tạo sử dụng phương pháp học bán giám sát là lan truyền ngược và giải thuật di truyền
\item Đề xuất phương pháp tính lưu lượng kênh truyền vào giờ bận mở rộng không chỉ với các cuộc gọi mà còn tính toán cả với các tin nhắn và các dịch vụ giá trị gia tăng
\item Thử nghiệm các mô hình vừa khảo sát với bài toán dự báo lưu lượng kênh truyền. Với mô hình tốt nhất đã thử nghiệm (mạng nơ ron nhân tạo có 3 nút đầu vào, 2 nút ẩn và 1 nút đầu ra sử dụng giải thuật di truyền tối ưu bộ trọng số) cho trung bình tỉ lệ lỗi so với giá trị của chuỗi chỉ khoảng $sMAPE=2.975\%$.Với dữ liệu có giá trị trung bình $\mu = 192.9621$ thì tỉ lệ lỗi như vậy chấp nhận được. Mô hình đề xuất hoàn toàn có thể sử dụng được. Ta cũng đạt được kết quả với mô hình $ARIMA(0,1,2)(2,0,2)_7$  cho $sMAPE = 5.349\%$
\end{itemize}

\section{Khó khăn}
Trong quá trình thực hiện khóa luận, tồn tại một số khó khăn sau:
\begin{itemize}
\item Dữ liệu rất lớn (350GB) nhưng đa số các trường thuộc tính mập mờ, không có đặc tả dữ liệu hoàn chỉnh. Điều đó tạo rất nhiều khó khăn trong việc xác định phương pháp khai phá và các mô hình phù hợp cho dữ liệu.
\item Mô hình ARIMA thực hiện rất nhiều bước trong thủ tục với rất nhiều tiêu chuẩn khác nhau. Cần chọn ra những tiêu chuẩn phù hợp nhất để áp dụng cho mô hình. Ngoài ra, việc dự đoán mô hình thường phải trải qua rất nhiều bước lặp lại nhưng không có một phương pháp cụ thể nào xác định được mô hình tối ưu nhất. 
\end{itemize}

\section{Hướng nghiên cứu tiếp theo}
Bài toán dự đoán mở ra rất nhiều hướng nghiên cứu mới với khả năng áp dụng nhiều mô hình biến đổi khác nhau. Nghiên cứu tiếp theo là làm thế nào để xác định tối ưu các tham số trong cả 2 mô hình với số bước thử lặp đi lặp lại ít nhất. Ngoài ra, chúng tôi đang tiếp tục thử nghiệm một số mô hình dự báo khác như mô hình Markov ẩn, mô hình Bayesian, phương pháp k-người láng giềng gần nhất để áp dụng cho bài toán dự báo. 